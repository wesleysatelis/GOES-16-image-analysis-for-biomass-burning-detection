\documentclass[11pt,brazil,]{article}
%DIF LATEXDIFF DIFFERENCE FILE
%DIF DEL relatorio_rev0_.tex            Mon Sep 14 00:27:20 2020
%DIF ADD relatorio_rev_guilherme_.tex   Mon Sep 14 00:23:49 2020
\usepackage{lmodern}
\usepackage{amssymb,amsmath}
\usepackage{ifxetex,ifluatex}
\usepackage{fixltx2e} % provides \textsubscript
\ifnum 0\ifxetex 1\fi\ifluatex 1\fi=0 % if pdftex
  \usepackage[T1]{fontenc}
  \usepackage[utf8]{inputenc}
\else % if luatex or xelatex
  \ifxetex
    \usepackage{mathspec}
  \else
    \usepackage{fontspec}
  \fi
  \defaultfontfeatures{Ligatures=TeX,Scale=MatchLowercase}
\fi
% use upquote if available, for straight quotes in verbatim environments
\IfFileExists{upquote.sty}{\usepackage{upquote}}{}
% use microtype if available
\IfFileExists{microtype.sty}{%
\usepackage{microtype}
\UseMicrotypeSet[protrusion]{basicmath} % disable protrusion for tt fonts
}{}
\usepackage[left=2.3cm, right=2.3cm, top=2cm, bottom=2cm]{geometry}
\usepackage{hyperref}
\hypersetup{unicode=true,
            pdftitle={Uso de imagens de satélites para detecção de queimadas nas áreas canavieiras de São Paulo},
            pdfauthor={Wesley R. da Silva Satelis;  Jurandir Zullo Jr., Renata R. V. Gonçalves e Guilherme V. N. Ludwig},
            pdfborder={0 0 0},
            breaklinks=true}
\urlstyle{same}  % don't use monospace font for urls
\ifnum 0\ifxetex 1\fi\ifluatex 1\fi=0 % if pdftex
  \usepackage[shorthands=off,main=brazil]{babel}
\else
  \usepackage{polyglossia}
  \setmainlanguage[]{brazil}
\fi
\usepackage{graphicx,grffile}
\makeatletter
\def\maxwidth{\ifdim\Gin@nat@width>\linewidth\linewidth\else\Gin@nat@width\fi}
\def\maxheight{\ifdim\Gin@nat@height>\textheight\textheight\else\Gin@nat@height\fi}
\makeatother
% Scale images if necessary, so that they will not overflow the page
% margins by default, and it is still possible to overwrite the defaults
% using explicit options in \includegraphics[width, height, ...]{}
\setkeys{Gin}{width=\maxwidth,height=\maxheight,keepaspectratio}
\IfFileExists{parskip.sty}{%
\usepackage{parskip}
}{% else
\setlength{\parindent}{0pt}
\setlength{\parskip}{6pt plus 2pt minus 1pt}
}
\setlength{\emergencystretch}{3em}  % prevent overfull lines
\providecommand{\tightlist}{%
  \setlength{\itemsep}{0pt}\setlength{\parskip}{0pt}}
\setcounter{secnumdepth}{5}
% Redefines (sub)paragraphs to behave more like sections
\ifx\paragraph\undefined\else
\let\oldparagraph\paragraph
\renewcommand{\paragraph}[1]{\oldparagraph{#1}\mbox{}}
\fi
\ifx\subparagraph\undefined\else
\let\oldsubparagraph\subparagraph
\renewcommand{\subparagraph}[1]{\oldsubparagraph{#1}\mbox{}}
\fi

%%% Use protect on footnotes to avoid problems with footnotes in titles
\let\rmarkdownfootnote\footnote%
\def\footnote{\protect\rmarkdownfootnote}

%%% Change title format to be more compact
\usepackage{titling}

% Create subtitle command for use in maketitle
\providecommand{\subtitle}[1]{
  \posttitle{
    \begin{center}\large#1\end{center}
    }
}

\setlength{\droptitle}{-2em}

  \title{\centering \LARGE Uso de imagens de satélites para detecção de queimadas
nas áreas canavieiras de São Paulo}
    \pretitle{\vspace{\droptitle}\centering\huge}
  \posttitle{\par}
    \author{\normalsize Wesley R. da Silva Satelis \\ \centering \normalsize \textbf{Orientação:} Jurandir Zullo Jr., Renata
R. V. Gonçalves e Guilherme V. N. Ludwig}
    \preauthor{\centering\large\emph}
  \postauthor{\par}
    \date{}
    \predate{}\postdate{}
  
\usepackage{booktabs}
\usepackage{longtable}
\usepackage{array}
\usepackage{multirow}
\usepackage{wrapfig}
\usepackage{float}
\usepackage{colortbl}
\usepackage{pdflscape}
\usepackage{tabu}
\usepackage{threeparttable}
\usepackage{threeparttablex}
\usepackage[normalem]{ulem}
\usepackage{makecell}

\usepackage{subcaption}
\usepackage{tikz}
\usepackage{caption}
\usepackage{float}
\usepackage{booktabs}
\usepackage{longtable}
\usepackage{array}
\usepackage{multirow}
\usepackage{wrapfig}
\usepackage{float}
\usepackage{colortbl}
\usepackage{pdflscape}
\usepackage{tabu}
\usepackage{threeparttable}
\usepackage{threeparttablex}
\usepackage[normalem]{ulem}
\usepackage{makecell}
\floatplacement{figure}{H}
%DIF PREAMBLE EXTENSION ADDED BY LATEXDIFF
%DIF UNDERLINE PREAMBLE %DIF PREAMBLE
\RequirePackage[normalem]{ulem} %DIF PREAMBLE
\RequirePackage{color}\definecolor{RED}{rgb}{1,0,0}\definecolor{BLUE}{rgb}{0,0,1} %DIF PREAMBLE
\providecommand{\DIFaddtex}[1]{{\protect\color{blue}\uwave{#1}}} %DIF PREAMBLE
\providecommand{\DIFdeltex}[1]{{\protect\color{red}\sout{#1}}}                      %DIF PREAMBLE
%DIF SAFE PREAMBLE %DIF PREAMBLE
\providecommand{\DIFaddbegin}{} %DIF PREAMBLE
\providecommand{\DIFaddend}{} %DIF PREAMBLE
\providecommand{\DIFdelbegin}{} %DIF PREAMBLE
\providecommand{\DIFdelend}{} %DIF PREAMBLE
%DIF FLOATSAFE PREAMBLE %DIF PREAMBLE
\providecommand{\DIFaddFL}[1]{\DIFadd{#1}} %DIF PREAMBLE
\providecommand{\DIFdelFL}[1]{\DIFdel{#1}} %DIF PREAMBLE
\providecommand{\DIFaddbeginFL}{} %DIF PREAMBLE
\providecommand{\DIFaddendFL}{} %DIF PREAMBLE
\providecommand{\DIFdelbeginFL}{} %DIF PREAMBLE
\providecommand{\DIFdelendFL}{} %DIF PREAMBLE
%DIF HYPERREF PREAMBLE %DIF PREAMBLE
\providecommand{\DIFadd}[1]{\texorpdfstring{\DIFaddtex{#1}}{#1}} %DIF PREAMBLE
\providecommand{\DIFdel}[1]{\texorpdfstring{\DIFdeltex{#1}}{}} %DIF PREAMBLE
\newcommand{\DIFscaledelfig}{0.5}
%DIF HIGHLIGHTGRAPHICS PREAMBLE %DIF PREAMBLE
\RequirePackage{settobox} %DIF PREAMBLE
\RequirePackage{letltxmacro} %DIF PREAMBLE
\newsavebox{\DIFdelgraphicsbox} %DIF PREAMBLE
\newlength{\DIFdelgraphicswidth} %DIF PREAMBLE
\newlength{\DIFdelgraphicsheight} %DIF PREAMBLE
% store original definition of \includegraphics %DIF PREAMBLE
\LetLtxMacro{\DIFOincludegraphics}{\includegraphics} %DIF PREAMBLE
\newcommand{\DIFaddincludegraphics}[2][]{{\color{blue}\fbox{\DIFOincludegraphics[#1]{#2}}}} %DIF PREAMBLE
\newcommand{\DIFdelincludegraphics}[2][]{% %DIF PREAMBLE
\sbox{\DIFdelgraphicsbox}{\DIFOincludegraphics[#1]{#2}}% %DIF PREAMBLE
\settoboxwidth{\DIFdelgraphicswidth}{\DIFdelgraphicsbox} %DIF PREAMBLE
\settoboxtotalheight{\DIFdelgraphicsheight}{\DIFdelgraphicsbox} %DIF PREAMBLE
\scalebox{\DIFscaledelfig}{% %DIF PREAMBLE
\parbox[b]{\DIFdelgraphicswidth}{\usebox{\DIFdelgraphicsbox}\\[-\baselineskip] \rule{\DIFdelgraphicswidth}{0em}}\llap{\resizebox{\DIFdelgraphicswidth}{\DIFdelgraphicsheight}{% %DIF PREAMBLE
\setlength{\unitlength}{\DIFdelgraphicswidth}% %DIF PREAMBLE
\begin{picture}(1,1)% %DIF PREAMBLE
\thicklines\linethickness{2pt} %DIF PREAMBLE
{\color[rgb]{1,0,0}\put(0,0){\framebox(1,1){}}}% %DIF PREAMBLE
{\color[rgb]{1,0,0}\put(0,0){\line( 1,1){1}}}% %DIF PREAMBLE
{\color[rgb]{1,0,0}\put(0,1){\line(1,-1){1}}}% %DIF PREAMBLE
\end{picture}% %DIF PREAMBLE
}\hspace*{3pt}}} %DIF PREAMBLE
} %DIF PREAMBLE
\LetLtxMacro{\DIFOaddbegin}{\DIFaddbegin} %DIF PREAMBLE
\LetLtxMacro{\DIFOaddend}{\DIFaddend} %DIF PREAMBLE
\LetLtxMacro{\DIFOdelbegin}{\DIFdelbegin} %DIF PREAMBLE
\LetLtxMacro{\DIFOdelend}{\DIFdelend} %DIF PREAMBLE
\DeclareRobustCommand{\DIFaddbegin}{\DIFOaddbegin \let\includegraphics\DIFaddincludegraphics} %DIF PREAMBLE
\DeclareRobustCommand{\DIFaddend}{\DIFOaddend \let\includegraphics\DIFOincludegraphics} %DIF PREAMBLE
\DeclareRobustCommand{\DIFdelbegin}{\DIFOdelbegin \let\includegraphics\DIFdelincludegraphics} %DIF PREAMBLE
\DeclareRobustCommand{\DIFdelend}{\DIFOaddend \let\includegraphics\DIFOincludegraphics} %DIF PREAMBLE
\LetLtxMacro{\DIFOaddbeginFL}{\DIFaddbeginFL} %DIF PREAMBLE
\LetLtxMacro{\DIFOaddendFL}{\DIFaddendFL} %DIF PREAMBLE
\LetLtxMacro{\DIFOdelbeginFL}{\DIFdelbeginFL} %DIF PREAMBLE
\LetLtxMacro{\DIFOdelendFL}{\DIFdelendFL} %DIF PREAMBLE
\DeclareRobustCommand{\DIFaddbeginFL}{\DIFOaddbeginFL \let\includegraphics\DIFaddincludegraphics} %DIF PREAMBLE
\DeclareRobustCommand{\DIFaddendFL}{\DIFOaddendFL \let\includegraphics\DIFOincludegraphics} %DIF PREAMBLE
\DeclareRobustCommand{\DIFdelbeginFL}{\DIFOdelbeginFL \let\includegraphics\DIFdelincludegraphics} %DIF PREAMBLE
\DeclareRobustCommand{\DIFdelendFL}{\DIFOaddendFL \let\includegraphics\DIFOincludegraphics} %DIF PREAMBLE
%DIF END PREAMBLE EXTENSION ADDED BY LATEXDIFF

\begin{document}
\maketitle
\begin{abstract}
This is sample text for abstract. Generally speaking, I would like to
show keywords list below an abstract (as in case of the linked example).

\par

\textbf{Palavras-chave:} séries temporais, sensoriamento remoto,
goes-16, pontos de mudança, agricultura, protocolo ambiental
\end{abstract}

\renewcommand{\figurename}{Figura}
\renewcommand{\tablename}{Tabela}
\captionsetup[figure]{font=small}

\hypertarget{introducao}{%
\section{Introdução}\label{introducao}}

A cultura da cana-de-açúcar possui função cada vez mais estratégica na
\DIFdelbegin \DIFdel{econômia }\DIFdelend \DIFaddbegin \DIFadd{economia }\DIFaddend do país devido ao interesse em conciliar preocupações
relacionadas ao meio ambiente e a utilização de combustíveis fósseis,
que considera o balanço do lançamento de carbono na atmosfera e suas
contribuições no aquecimento global.

Com propósito de reduzir a emissão de gases de efeito estufa (GEEs) na
atmosfera, vem ocorrendo a eliminação do emprego de fogo para despalha
da cana-de-açúcar, nas lavouras do estado de São Paulo. A mecanização da
colheita sem queima prévia evita a emissão de gases de efeito estufa e
beneficia o solo, pois deixa sobre ele a palha que antes era queimada e
o protege contra erosão, além de contribuir para o aumento de
fertilidade e teor de matéria orgânica (Conab
\protect\hyperlink{ref-conab}{2019}).

O mapeamento e monitoramento das lavouras de cana-de-açúcar, com e sem
pré-queima da palha, torna-se importante para avaliar a eficácia do
protocolo de intenções assinado em junho de 2007 pela Secretaria do Meio
Ambiente do Estado de São Paulo (SMA-SP) e a União da Indústria de
Cana-de-Açúcar (UNICA).

Novaes et al. (\protect\hyperlink{ref-protocloagro}{2011}) e Mello
(\protect\hyperlink{ref-orbitainpe}{2009}) afirmam que técnicas de
análise espacial de imagens de satélites são essenciais para o
mapeamento e monitoramento da colheita da cana-de-açúcar com queima da
palha. Rudorff et al. (\protect\hyperlink{ref-rudorff2010studies}{2010})
e Gonçalves et al.~(2012a e 2012b) confirmaram que imagens de satélites
são eficientes para auxiliar a avaliação de características importantes
do cultivo da cana-de-açúcar, proporcionando resultados relevantes para
o debate sobre a produção sustentável de etanol. (Adami et al.
\protect\hyperlink{ref-sugarcanesouth}{2012}), avaliando a precisão do
mapeamento temático da cana-de-açúcar por meio de imagens de satélites,
chegaram a estimativas precisas das áreas de cana-de-açúcar para fins de
estatísticas agrícolas empregradas no monitoramento da expansão de
cultura no país.

Este projeto de pesquisa faz uso das imagens do satélite meteorológico
GOES-16 referentes ao estado de São Paulo, empregando técnicas de
análise de pontos de mudança em séries temporais.

As áreas de plantio serão modeladas como séries temporais espacialmente
dependentes entre si, por um processo de média móvel (ARMA) (\DIFaddbegin \DIFadd{uma
referência recente é }\DIFaddend Morettin e Toloi
\protect\hyperlink{ref-morettin2006analise}{2006}), incluindo um
componente de pontos de mudança (Aminikhanghahi e Cook
\protect\hyperlink{ref-aminikhanghahi2017survey}{2017}). O interesse
principal é identificar quantos pontos de mudança foram observados, em
quais momentos e qual a mudança resultante e sugerir um método de
detecção de áreas com queima de biomassa refente à cana-de-açúcar.

\hypertarget{objetivos}{%
\section{Objetivos}\label{objetivos}}

\hypertarget{objetivo-geral}{%
\subsection{Objetivo geral}\label{objetivo-geral}}

O projeto tem, como objetivo geral, monitorar a colheita de
cana-de-açúcar da safra 2019/2020, por meio de imagens do satélite
GOES-16, quantificando as áreas colhidas com e sem queima.

\hypertarget{objetivos-especificos}{%
\subsection{Objetivos específicos}\label{objetivos-especificos}}

Os objetivos específicos são: i) Analisar a variação dos valores do NDVI
da cana-de-açúcar no estado de São Paulo no período de colheita da
cultura; ii) Selecionar as áreas colhidas de cana-de-açúcar e verificar
se houve ou não colheita por queima; iii) Quantificar a área colhida de
cana-de-açúcar por queima e não queima.

\hypertarget{materiais-e-metodos}{%
\section{Materiais e métodos}\label{materiais-e-metodos}}

\hypertarget{indices-de-vegetacao-e-queimada}{%
\subsection{Índices de vegetação e
queimada}\label{indices-de-vegetacao-e-queimada}}

O espectro eletromagnético do GOES-16 é dividido em 16 intervalos, com
comprimentos de onda classificados entre visível, infravermelho próximo
e infra-vermelho. Neste projeto, estes intervalos serão denominados
faixas e denotados por \(\rho\).

Foram calculados índices de vegetação e queimada a fim de acompanhar a
cultura de cana e avaliar áreas onde houve pré-queima, resultando em uma
imagem completa do estado de São Paulo a cada faixa recebida do
satélite. Para tal, foi utilizado o Índice de Vegetação por Diferença
Normalizada (NDVI), definido por Rouse et al.
(\protect\hyperlink{ref-rouse1974monitoring}{1974}) e calculado por
\DIFdelbegin \DIFdel{,
}\DIFdelend 

\[ NDVI=\frac{\rho_{III} - \rho_{II}}{\rho_{III} + \rho_{II}}\DIFaddbegin \DIFadd{, }\DIFaddend \]

em que \DIFdelbegin \DIFdel{, }\DIFdelend \(\rho_{III}\) é o fator de refletância no infravermelho próximo
\((0,86 \mu m)\) e \(\rho_{II}\) no vermelho \((0,64 \mu m)\). Os
valores no NDVI variam entre -1,0 e 1,0, sendo maior quanto maior for a
diferença entre o fator de refletância no infravermelho próximo e no
vermelho. Valores próximos a zero correspondem a superfícies sem
vegetação.

O Índice de Queima Normalizada (NBR) (García e Caselles
\protect\hyperlink{ref-garcia1991mapping}{1991}), foi utilizado na
avaliação de áreas de queimadas. Uma vez que ele evidencia cicatrizes em
áreas de vegetação, ou seja áreas de vegetação onde houve queima de
biomassa. O NBR é calculado por
\DIFdelbegin \DIFdel{,
}\DIFdelend 

\[ NBR=\frac{\rho_{III} - \rho_{VI}}{\rho_{III} + \rho_{VI}}\DIFaddbegin \DIFadd{, }\DIFaddend \]

em que \DIFdelbegin \DIFdel{, }\DIFdelend \(\rho_{VI}\) é o fator de refletância no infravermelho de onda
curta \((2,24 \mu m)\) e \(\rho_{III}\) é o mesmo utilizado no cálculo
de NDVI. Valores negativos ou relativamente próximos a zero corresponde
a áreas com vegetação queimada.

\hypertarget{extracao-de-dados-e-georeferenciamento}{%
\subsection{Extração de dados e
georeferenciamento}\label{extracao-de-dados-e-georeferenciamento}}

Em 2018, com recursos da Financiadora de Estudos e Projetos (Finep), o
Centro de Pesquisas Meteorológicas e Climáticas Aplicadas à Agricultura
(CEPAGRI) adiquiriu e instalou um sistema de recepção e processamento de
imagens do satélite meteorológico GOES-16.

O sistema possui um software específico para o cálculo de índices a
partir de diferentes faixas espectrais. Entretanto, buscando maior
controle sobre a qualidade, os cálculos foram feitos diretamente com
códigos implementados na linguagem de computação estatística, R. Assim,
optou-se por fazer uso do software do sistema somente para recorte da
área de interesse e seleção das faixas espectrais, minimizando o tamanho
em disco e, consequentemente, tempo de processamento.

Os scans recebidos abrangem todo o ocidente e o pré processamento tem o
objetivo de extrair somente a região de estudo. O software do sistema de
recepção retorna uma matriz para cada faixa espectral, compreendendo
todo o estado de São Paulo e com resolução espacial de aproximandamente
500 m por pixel.

Os dados pré-processados estão dispostos em formato de texto sem
qualquer referenciamento disponível. O georeferenciamento foi feito
diretamente sobrepondo uma camada contendo as coordenadas que referencia
os índices pelas fronteiras do estado e dá a cada pixel uma latitude e
longitude.

As localizações das cultura canavieiras utilizadas neste projeto provêm
de um estudo feito por Aguiar et al.
(\protect\hyperlink{ref-aguiar2011remote}{2011}) no Institudo Nacional
de Pesquisas Espaciais (INPE), que também teve o objetivo de monitorar o
cumprimento do protocolo ambiental no cultivo de cana-de-açúcar no
estado de São Paulo. Além destas, foram extraídas localizações de focos
de incêndio reportadas pelo programa de monitoramento de queimadas,
também desenvolvido pelo INPE (Instituto Nacional de Pesquisas
Espaciais, \protect\hyperlink{ref-inpe}{{[}s.d.{]}}), selecionando
municípios com plantações de tamanho considerável de cana. As técnicas
discutidas nas sessões a seguir serão aplicadas nestes pontos, a fim de
criar um método rasoável de classificação de queimas em vegetação. Este
conjunto de dados será denominado conjunto de testes.

\hypertarget{reducao-de-efeitos-atmosfericos}{%
\subsection{Redução de efeitos
atmosféricos}\label{reducao-de-efeitos-atmosfericos}}

Técnicas de redução de efeitos atmosféricos são encaradas neste estudo
como métodos de processamento de sinais em séries temporais e objetivam
minimizar ruídos causados por fatores externos, como contaminações por
nuvens, \DIFdelbegin \DIFdel{âgulos }\DIFdelend \DIFaddbegin \DIFadd{ângulos }\DIFaddend de luz solar, efeitos de sombra, efeitos de aerosol e
vapor de água e refletância direcional.

O procedimento de Composição de Máximo Valor (MVC) (Holben
\protect\hyperlink{ref-holben1986characteristics}{1986}), avalia valores
em intervalos de tempo predeterminados e mantém os pixels com maior
valor. Por fim, tem-se uma imagem resultante da combinação dos valores
máximos de todas as imagens na janela de tempo fixado. Neste estudo,
essa técnica foi aplicada às séries temporais com observações entre
9h-11h e 13h-15h UTC−3 de cada dia. A escolha dos intervalos se dá pelo
fato de a qualidade do resultado depender fortemente da quantidade de
ruído presente no intervalo e ambos os índices apresentarem menor
contaminção em horários ao redor do meio-dia. Vale ressaltar que esta
técnica diminui consideravelmente a resolução temporal dos dados, indo
de uma observação a cada 15 minutos para duas observações por dia e,
consequentemente, reduz a possibilidade de detectar mudanças de curto
prazo.

Visando manter a resolução temporal, foi aplicado um filtro de \DIFdelbegin \DIFdel{madiana
}\DIFdelend \DIFaddbegin \DIFadd{mediana
}\DIFaddend adaptativo definido por Schettlinger, Fried, e Gather
(\protect\hyperlink{ref-medianfilter}{2009}), em que o tamanho da janela
é adapatado para os dados da janela atual por um teste de ajuste do
sinal estimado mais recente, mantendo a mediana de cada janela móvel.

O filtro \DIFdelbegin \DIFdel{aplica localmente o modelo
,
}\DIFdelend \DIFaddbegin \DIFadd{é construído localmente com base no modelo
}\DIFaddend 

\[Y_t = Y_{t-j} + (t-j)\beta + \epsilon_{j}, j = 1, 2, ..., n_t\DIFaddbegin \DIFadd{,}\DIFaddend \]

isto é, \DIFdelbegin \DIFdel{incrementando na série }\DIFdelend \DIFaddbegin \DIFadd{em que as últimas \(n_t\) observações em um instante de tempo
arbitrário \(t\) são incrementos lineares de }\DIFaddend \(\beta\) unidades \DIFdelbegin \DIFdel{a }\DIFdelend \DIFaddbegin \DIFadd{sobre
valores anteriores, para }\DIFaddend cada uma unidade de tempo.
\DIFdelbegin \DIFdel{\(n_t\) é variável.
}\DIFdelend 

Seja uma série definida em \DIFdelbegin \DIFdel{\(Y_1, Y_2, ..., Y_t\) no }\DIFdelend \DIFaddbegin \DIFadd{\(Y_1, Y_2, ..., Y_t.\) No }\DIFaddend \(t\)-ésimo
instante, queremos encontrar o valor filtrado em \(Y_t\) dado por
\(Y_t^{*}\). \DIFdelbegin \DIFdel{Fixa-se \(n_t\) e os dados passam a ser a janela
}\DIFdelend \DIFaddbegin \DIFadd{Para um \(n_t,\) os dados que irão compor o valor filtrado
são }\DIFaddend \(Y_t, Y_{t - 1}, ..., Y_{t - n_{t + 1}}\). Obtemos o estimador
\(\hat{\beta} = Med_j\{Y_{t-j}, Y_{t-j-1}\}\), o incremento mediano
sobre \(Y\) por uma unidade de tempo, para a janela de \(n_t\)
observações. Assumimos que dentro da janela, a série está incrementando
beta unidades por uma unidade de tempo e \(\hat{\beta}\) é um estimador
robusto de beta.

Como a série assume um efeito linear local, a mediana dos incrementos
lineares das últimas \(n_t\) observações é uma previsão robusta do sinal
de \(Y_t\), \(Y_t^{*} = Med_j\{Y_{t - j - 1} + (n-j)\beta\}\). \DIFaddbegin \DIFadd{A escolha
de \(n_t\) é adaptativa e está descrita em Killick, Fearnhead, e Eckley
(}\protect\hyperlink{ref-killick2012optimal}{2012}\DIFadd{).
}\DIFaddend 

A técnica foi aplicada \DIFdelbegin \DIFdel{independentemente do MVC, }\DIFdelend utilizando o pacote \DIFdelbegin \DIFdel{descrito em }\DIFdelend \DIFaddbegin \DIFadd{implementado por }\DIFaddend Fried,
Schettlinger, e Borowski (\protect\hyperlink{ref-robfilter}{2019})\DIFaddbegin \DIFadd{, e
será comparada ao MVC}\DIFaddend .

\hypertarget{pontos-de-mudanca}{%
\subsection{Pontos de mudança}\label{pontos-de-mudanca}}

\DIFdelbegin \DIFdel{Neste trabalho de pesquisa a }\DIFdelend \DIFaddbegin \DIFadd{A }\DIFaddend identificação de pontos de mudança em séries temporais do NDVI e NBR
\DIFdelbegin \DIFdel{, }\DIFdelend tem o objetivo específico de \DIFdelbegin \DIFdel{descriminar
}\DIFdelend \DIFaddbegin \DIFadd{discriminar }\DIFaddend intervalos em que ocorrem
mudanças nas propriedades estatísticas, \(\mu\) e \(\sigma\), causadas
pela pré-queima da palha de cana-de-açúcar.

Seja \(\{Y_{1, t}\}\) a série correspondente a um pixel do conjunto de
testes, um ponto de mudança ocorre quando existe um tempo \(t_{\tau}\)
\(\epsilon\) \(\{0, ..., t-1\}\) em que as propriedades estatísticas de
\(\{y_{1, 0}, ..., y_{1, \tau}\}\) e
\(\{y_{1, \tau+1}, ..., y_{1, t}\}\) se diferem de alguma forma. A
detecção de um ponto de mudança pode ser encarada como um teste de
hipóteses. A hipótese nula, \(H_0\), corresponde a nenhum ponto de
mudança \((m=0)\) e a hipótese alternativa, \(H_1\), a um ponto de
mudança \((m=1)\). \DIFaddbegin \DIFadd{Se rejeitamos a hipótese nula, o conjunto de dados é
segmentado no ponto de mudança, e iteramos o procedimento, corrigindo os
testes para evitar descobertas falsas.
}\DIFaddend 

A estatística associada ao teste de hipóteses proposto é de razão de
verossimilhanças e faz uso da função de log-verossimilhança sob ambas
hipóteses, \(H_0\) e \(H_1\). Sob a hipótese alternativa, considere um
modelo com um ponto de mudança em \(t_{\tau}\), em que
\(t_{\tau} \in {1, 2, ..., n-1}\). Assim, a função de
log-verossimilhança é dada por,

\[ML\left(\tau_{1}\right)=\log p\left(y_{1: \tau_{1}} \mid \hat{\theta}_{1}\right)+\log p\left(y_{\left(\tau_{1}+1\right): n} \mid \hat{\theta}_{2}\right).\]

Levando em conta a natureza discreta da localização de pontos de
mudança, o valor log-verossimilhança sob a hipótese alternativa é
simplesmente \(\max_{\tau_{1}}ML\left(\tau_{1}\right)\), em que o valor
máximo é tomado sobre todos os pontos de mudança possíveis. Assim, a
estatística do teste é,

\[\lambda=2\left[\max _{\tau_{1}} M L\left(\tau_{1}\right)-\log p\left(y_{1: n} \mid \hat{\theta}\right)\right].\]

O teste envolve a escolha de um limite, \(c\), tal que a hipótese nula é
rejeitada se \(\lambda > c\). Se rejeitamos a hipótese nula, ou seja,
detectamos um ponto de mudança, então estimamos sua posição como
\(\hat{\tau_{1}}\).

O interesse principal é identificar quantos pontos de mudança foram
observados no histórico (\(m\)), em quais momentos
(\(t_1, t_2, \ldots, t_m\)) e qual a mudança resultante
(\(\alpha_1, \ldots, \alpha_m\)) (Killick, Fearnhead, e Eckley
\protect\hyperlink{ref-killick2012optimal}{2012}). \DIFdelbegin %DIFDELCMD < 

%DIFDELCMD < %%%
\DIFdelend A abordagem mais
comum na detecção de múltiplos pontos de mudança na literatura é
minimizando,

\[\sum_{i=1}^{m+1}\left[\mathcal{C}\left(y_{\left(\tau_{i-1}+1\right): \tau_{i}}\right)\right]+\beta f(m)\]

em que, \(\mathcal{C}\) é a função de custo para um segmento, por
exemplo a função de log-verossimilhança e \(\beta f(m)\) uma função
penalidade para prevenir sobreajustes, uma versão para múltiplos pontos
de mudança do limite \(c\) citado anteriormente. Na prática, a escolha
mais comum é uma que seja linear no número de pontos de mudança, isto é,
\(\beta f(m) = \beta m\). Temos como exemplos comuns as penalidades
\emph{Akaike's information crtierion} (AIC), em que \(\beta = 2p\) e
\emph{Bayesian information criterion} (BIC), em que \(\beta = plog(n)\)
com p sendo o número de parâmentros adicionais introduzidos pela adição
de um ponto de mudança.

O método de busca de pontos de mudança aplicado foi o de segmentação
binária, empregado utilizando o pacote descrito por Killick e Eckley
(\protect\hyperlink{ref-killick2014changepoint}{2014}). Em suma a
segmentação binária torna qualquer método para um ponto de mudança em um
de multiplos pontos, repetindo-se iterativamente em diferentes
subconjuntos da série.

Foram estimados pontos de mudança em média, \(\mu\), e variância,
\(\sigma^2\), conjuntamente e apesar de terem sido estimadas mudanças em
variância, estas serão referidas no decorrer deste texto na forma de
desvio padrão \(\sqrt{\sigma^2}\) e denotadas por \(\sigma\).

\hypertarget{resultados}{%
\section{Resultados}\label{resultados}}

Os índices de Vegetação por Diferença Normalizada e de Queima
Normalizada calculados para todo o território de São Paulo estão
representados na Figura \ref{indices}. Todas as imagens recebidas
passaram pelo mesmo procedimento, possibilitando a obtenção das séries
de qualquer pixel contido nas imagens.

\begin{figure}[H]

{\centering \DIFdelbeginFL %DIFDELCMD < \includegraphics{relatorio_rev0__files/figure-latex/unnamed-chunk-1-1} 
%DIFDELCMD < %%%
\DIFdelendFL \DIFaddbeginFL \includegraphics{relatorio_rev_guilherme__files/figure-latex/unnamed-chunk-1-1} 
\DIFaddendFL 

}

\caption{\label{indices} Índice de Vegetação por Diferença Normalizada (NDVI) à esquerda e Índice de Queima Normalizada (NBR) à direita, georeferenciados pelas fronteiras do estado de São Paulo}\label{fig:unnamed-chunk-1}
\end{figure}

Nas Figuras \ref{mvc} e \ref{med}, as linhas pretas mostram os
resultados das reduções de ruídos pelos métodos de composição de valor
máximo e pelo filtro de mediana com janela móvel adaptativa,
respectivamente. Como esperado, ambos estão dentro do intervalo teórico
de (-1,0 e 1,0).

A fidelidade do método redução de ruído pelo filtro de médiana
mostrou-se eficiente, uma vez que se assemelha ao método mais fiel, de
composição por valor máximo. Cumprindo com o objetivo de manter a
resolução temporal original dos dados sem perda de fidelidade.

Os segmentos em vermelho nas Figuras \ref{mvc} e \ref{med} representam
os intervalos em que as médias e desvios padrões se diferem, ou seja,
intervalos dos pontos de mudança detectados. A linha tracejada em
laranja é hora exata em que a queimada foi detectada pelo sistema de
monitoramento de queimadas do INPE.

Nas Tabelas \ref{ndvi_mvc} e \ref{nbr_mvc} estão as estimativas das
médias e desvios padrões, para cada segmento de mudança das séries
resultantes por composição de valor máximo, bem como as datas e horas
que formam os intervalos em que foram detectadas. Analogamente as
Tabelas \ref{ndvi_med} e \ref{nbr_med} trazem a mesma informação para as
séries resultantes do filtro de mediana com janela móvel adaptativa.

\begin{figure}[H]

{\centering \DIFdelbeginFL %DIFDELCMD < \includegraphics{relatorio_rev0__files/figure-latex/unnamed-chunk-2-1} 
%DIFDELCMD < %%%
\DIFdelendFL \DIFaddbeginFL \includegraphics{relatorio_rev_guilherme__files/figure-latex/unnamed-chunk-2-1} 
\DIFaddendFL 

}

\caption{\label{mvc} Em vermelho, segmentos de mudança em média e desvio padrão detectados na série resultante da composição por valor máximo (MVC) de um pixel pertencente ao conjunto de testes do município de Sertãozinho. NDVI à esquerda e NBR à direita}\label{fig:unnamed-chunk-2}
\end{figure}

\begin{table}[t]

\caption{\label{tab:unnamed-chunk-3}\label{ndvi_mvc}Estimativas de pontos de mudança para o Índice de vegetação (NDVI) tratado por MVC do município de Sertãozinho}
\centering
\begin{tabular}{cccc}
\toprule
Data de início & Data de fim & Média & Desvio padrão\\
\midrule
2019-01-01 12:00:00 & 2019-04-14 18:00:00 & 0.462 & 0.239\\
2019-04-15 12:00:00 & 2019-07-14 17:00:00 & 0.530 & 0.148\\
2019-07-14 18:00:00 & 2019-09-18 17:00:00 & 0.364 & 0.116\\
2019-09-18 18:00:00 & 2019-11-15 18:00:00 & 0.249 & 0.119\\
2019-11-16 12:00:00 & 2019-12-24 14:00:00 & 0.366 & 0.203\\
2019-12-24 16:00:00 & 2019-12-31 18:00:00 & 0.708 & 0.029\\
\bottomrule
\end{tabular}
\end{table}

\begin{table}[t]

\caption{\label{tab:unnamed-chunk-3}\label{nbr_mvc}Estimativas de pontos de mudança para o Índice de queimada (NBR) tratado por MVC do município de Sertãozinho}
\centering
\begin{tabular}{cccc}
\toprule
Data de início & Data de fim & Média & Desvio padrão\\
\midrule
2019-01-01 12:00:00 & 2019-05-31 14:00:00 & 0.534 & 0.128\\
2019-05-31 16:00:00 & 2019-07-25 12:00:00 & 0.488 & 0.079\\
2019-07-25 13:00:00 & 2019-08-19 12:00:00 & 0.369 & 0.087\\
2019-08-19 13:00:00 & 2019-11-13 12:00:00 & 0.271 & 0.106\\
2019-11-13 13:00:00 & 2019-12-24 14:00:00 & 0.452 & 0.086\\
2019-12-24 16:00:00 & 2019-12-31 18:00:00 & 0.674 & 0.035\\
\bottomrule
\end{tabular}
\end{table}

\begin{figure}[H]

{\centering \DIFdelbeginFL %DIFDELCMD < \includegraphics{relatorio_rev0__files/figure-latex/unnamed-chunk-4-1} 
%DIFDELCMD < %%%
\DIFdelendFL \DIFaddbeginFL \includegraphics{relatorio_rev_guilherme__files/figure-latex/unnamed-chunk-4-1} 
\DIFaddendFL 

}

\caption{\label{med} Em vermelho, segmentos de mudança em média e desvio padrão detectados na série resultante do filtro por mediana de um pixel pertencente ao conjunto de testes do município de Sertãozinho. NDVI à esquerda e NBR à direita}\label{fig:unnamed-chunk-4}
\end{figure}

\begin{table}[t]

\caption{\label{tab:unnamed-chunk-5}\label{ndvi_med}Estimativas de pontos de mudança para o Índice de vegetação (NDVI) tratado pelo filtro de mediana móvel do município de Sertãozinho}
\centering
\begin{tabular}{cccc}
\toprule
Data de início & Data de fim & Média & Desvio padrão\\
\midrule
2019-01-01 11:00:00 & 2019-07-06 14:20:00 & 0.462 & 0.239\\
2019-07-06 14:30:00 & 2019-07-14 19:50:00 & 0.530 & 0.148\\
2019-07-14 20:00:00 & 2019-09-18 19:30:00 & 0.364 & 0.116\\
2019-09-18 19:40:00 & 2019-10-23 12:40:00 & 0.249 & 0.119\\
2019-10-23 12:50:00 & 2019-12-24 14:30:00 & 0.366 & 0.203\\
2019-12-24 14:40:00 & 2019-12-31 21:00:00 & 0.708 & 0.029\\
\bottomrule
\end{tabular}
\end{table}

\begin{table}[t]

\caption{\label{tab:unnamed-chunk-5}\label{nbr_med}Estimativas de pontos de mudança para o Índice de queimada (NBR) tratado pelo filtro de mediana móvel do município de Sertãozinho}
\centering
\begin{tabular}{cccc}
\toprule
Data de início & Data de fim & Média & Desvio padrão\\
\midrule
2019-01-01 11:00:00 & 2019-07-25 13:00:00 & 0.534 & 0.128\\
2019-07-25 13:10:00 & 2019-08-19 13:10:00 & 0.488 & 0.079\\
2019-08-19 13:20:00 & 2019-10-27 20:20:00 & 0.369 & 0.087\\
2019-10-27 20:30:00 & 2019-11-15 13:20:00 & 0.271 & 0.106\\
2019-11-15 13:30:00 & 2019-12-24 16:20:00 & 0.452 & 0.086\\
2019-12-24 16:30:00 & 2019-12-31 21:00:00 & 0.674 & 0.035\\
\bottomrule
\end{tabular}
\end{table}

\hypertarget{discussao}{%
\section{Discussão}\label{discussao}}

\newpage

\hypertarget{referencias}{%
\section*{Referências}\label{referencias}}
\addcontentsline{toc}{section}{Referências}

\hypertarget{refs}{}
\leavevmode\hypertarget{ref-sugarcanesouth}{}%
Adami, M., M. P. Mello, D. A. Aguiar, B. F. T. Rudorff, e A. F. Souza.
2012. ``A web platform development to perform thematic accuracy
assessment of sugarcane mapping in South-Central Brazil''. \emph{Remote
Sensing} 4 (10): 3201--14.

\leavevmode\hypertarget{ref-aguiar2011remote}{}%
Aguiar, D. A., B. F. T. Rudorff, W. F. Silva, M. Adami, e M. P. Mello.
2011. ``Remote sensing images in support of environmental protocol:
Monitoring the sugarcane harvest in São Paulo State, Brazil''.
\emph{Remote Sensing} 3 (12): 2682--2703.

\leavevmode\hypertarget{ref-aminikhanghahi2017survey}{}%
Aminikhanghahi, S., e D. J. Cook. 2017. ``A survey of methods for time
series change point detection''. \emph{Knowledge and information
systems} 51 (2): 339--67.

\leavevmode\hypertarget{ref-conab}{}%
Conab. 2019. ``Cana-de-açúcar: Acompanhamento da safra brasileira:
Cana-de-açúcar -- Safra 2018/19 - Terceiro levantamento''.
\url{www.conab.gov.br/info-agro/safras/cana}.

\leavevmode\hypertarget{ref-robfilter}{}%
Fried, Roland, Karen Schettlinger, e Matthias Borowski. 2019.
\emph{robfilter: Robust Time Series Filters}.
\url{https://CRAN.R-project.org/package=robfilter}.

\leavevmode\hypertarget{ref-garcia1991mapping}{}%
García, M. J. L., e V. Caselles. 1991. ``Mapping burns and natural
reforestation using Thematic Mapper data''. \emph{Geocarto
International} 6 (1): 31--37.

\leavevmode\hypertarget{ref-gonccalves2012relaccao}{}%
Gonçalves, R. R. V., J. Zullo, P. P. Coltri, A. M. H. Avila, B. F.
Amaral, E. B. M. de Sousa, e L. A. S. Romani. 2012. ``Relação entre o
índice EVI e dados de precipitação nas áreas de plantio de
cana-de-açúcar na região central do Brasil.'' 4º Simpósio de
Geotecnologias no Pantanal - Geopantanal, Bonito, MS.: 2012b, Anais -
Geopantanal.

\leavevmode\hypertarget{ref-gonccalves2012evi}{}%
Gonçalves, R. R. V., J. Zullo, P. P. Coltri, e L. A. S. Romani. 2012.
``Evi's estimation to improve the monitoring of sugarcane using TRMM
satellite data''. In \emph{2012 IEEE International Geoscience and Remote
Sensing Symposium}, 6609--12. 2012a, IEEE.

\leavevmode\hypertarget{ref-holben1986characteristics}{}%
Holben, B. N. 1986. ``Characteristics of maximum-value composite images
from temporal AVHRR data''. \emph{International journal of remote
sensing} 7 (11): 1417--34.

\leavevmode\hypertarget{ref-inpe}{}%
Instituto Nacional de Pesquisas Espaciais, INPE --. {[}s.d.{]}. ``Portal
do Monitoramento de Queimadas e Incêndios Florestais''.
\url{http://queimadas.dgi.inpe.br/queimadas/portal}.

\leavevmode\hypertarget{ref-killick2014changepoint}{}%
Killick, R, e I Eckley. 2014. ``changepoint: An R package for
changepoint analysis''. \emph{Journal of statistical software} 58 (3):
1--19.

\leavevmode\hypertarget{ref-killick2012optimal}{}%
Killick, R., P. Fearnhead, e I. A. Eckley. 2012. ``Optimal detection of
changepoints with a linear computational cost''. \emph{Journal of the
American Statistical Association} 107 (500): 1590--8.

\leavevmode\hypertarget{ref-orbitainpe}{}%
Mello, M. P. 2009. ``Classificação espectro-temporal de imagens orbitais
para o mapeamento da colheita da cana-de-açúcar com queima da palha''.
Tese de doutorado, São José dos Campos, INPE.: MS dissertation, Nat.
Inst. Space Res., São José dos Campos, Brazil; Anais - SBSR.

\leavevmode\hypertarget{ref-morettin2006analise}{}%
Morettin, P. A., e C. M. Toloi. 2006. ``Análise de séries temporais''.
\emph{ABE--Projeto Fisher, Editora Edgar Blücher}.

\leavevmode\hypertarget{ref-protocloagro}{}%
Novaes, M. R. de, B. F. T. Rudorff, C. M. de Almeida, e D. A. de Aguiar.
2011. ``Análise espacial da redução da queima na colheita da
cana-de-açúcar: perspectivas futuras ao cumprimento do protocolo
agroambiental.'' 31 (3).

\leavevmode\hypertarget{ref-rouse1974monitoring}{}%
Rouse, J. W., R. H. Haas, J. A. Schell, e D. W. Deering. 1974.
``Monitoring vegetation systems in the Great Plains with ERTS''.
\emph{NASA special publication} 351: 309.

\leavevmode\hypertarget{ref-rudorff2010studies}{}%
Rudorff, B. F. T., D. A. de Aguiar, W. F. Silva, L. M. Sugawara, M.
Adami, e M. A. Moreira. 2010. ``Studies on the rapid expansion of
sugarcane for ethanol production in São Paulo State (Brazil) using
Landsat data''. \emph{Remote sensing} 2 (4): 1057--76.

\leavevmode\hypertarget{ref-medianfilter}{}%
Schettlinger, K., Roland Fried, e U. Gather. 2009. ``Real‐time signal
processing by adaptive repeated median filters''. \emph{International
Journal of Adaptive Control and Signal Processing} 24 (novembro):
346--62. \url{https://doi.org/10.1002/acs.1105}.


\end{document}
